\documentclass{beamer}



%\usepackage{beamerthemesplit}
\usetheme{Boadilla}
%\usetheme{default}
%\useinnertheme{rounded}

%\useoutertheme{shadow}
\usecolortheme{rose}
%\usefonttheme{serif}
\setbeamertemplate{navigation symbols}{}
\usetheme{Madrid}

\usepackage{amssymb,amsmath,amscd,amsfonts,amsthm,dsfont,color,graphicx}
\usepackage{amscd}
%\usepackage[numbers]{natbib}
% \usepackage[french]{babel}
%\usepackage[active]{srcltx}


\def\qd{\,{\mathchar'26\mkern-12mu d}}

% \date[]{}

 \newcommand\makebeamertitle{\frame{\maketitle}}%

 \AtBeginDocument{
   \let\origtableofcontents=\tableofcontents
   \def\tableofcontents{\@ifnextchar[{\origtableofcontents}{\gobbletableofcontents}}
   \def\gobbletableofcontents#1{\origtableofcontents}
 }
\numberwithin{equation}{section}
  \theoremstyle{plain}
  \newtheorem*{thm*}{\protect\theoremname}
  \theoremstyle{plain}
  \newtheorem*{cor*}{\protect\corollaryname}
 \theoremstyle{definition}
 \newtheorem*{defn*}{\protect\definitionname}
 \theoremstyle{plain}
\newtheorem*{lem*}{\protect\lemmaname}
  \theoremstyle{plain}
  \newtheorem*{rem*}{\protect\remarkname}
   \theoremstyle{definition}
 \newtheorem*{prop*}{\protect\propositionname}

\usetheme{Madrid}

\makeatother

  \providecommand{\corollaryname}{Corollary}
  \providecommand{\definitionname}{Definitioninition}
  \providecommand{\theoremname}{Theorem}
   \providecommand{\lemmaname}{Lemma}
   \providecommand{\remarkname}{Remark}
   \providecommand{\propositionname}{Proposition}
   
   
\newcommand{\Rl}{\mathbb{R}}
\newcommand{\Cplx}{\mathbb{C}}
\newcommand{\Itgr}{\mathbb{Z}}
\newcommand{\Ntrl}{\mathbb{N}}
\newcommand{\Circ}{\mathbb{T}}
\newcommand{\Sb}{\mathbb{S}}
\newcommand{\Disc}{\mathbb{D}}
\newcommand{\Aff}{\mathbb{A}}

% The Caligraphic alphabet
\newcommand{\Ac}{\mathcal{A}}
\newcommand{\Bc}{\mathcal{B}}
\newcommand{\Cc}{\mathcal{C}}
\newcommand{\Dc}{\mathcal{D}}
\newcommand{\Ec}{\mathcal{E}}
\newcommand{\Fc}{\mathcal{F}}
\newcommand{\Gc}{\mathcal{G}}
\newcommand{\Hc}{\mathcal{H}}
\newcommand{\Ic}{\mathcal{I}}
\newcommand{\Jc}{\mathcal{J}}
\newcommand{\Kc}{\mathcal{K}}
\newcommand{\Lc}{\mathcal{L}}
\newcommand{\Mv}{\mathcal{M}}
\newcommand{\Nv}{\mathcal{N}}
\newcommand{\Oc}{\mathcal{O}}
\newcommand{\Pc}{\mathcal{P}}
\newcommand{\Qc}{\mathcal{Q}}
\newcommand{\Rc}{\mathcal{R}}
\newcommand{\Sc}{\mathcal{S}}
\newcommand{\Tc}{\mathcal{T}}
\newcommand{\Uc}{\mathcal{U}}
\newcommand{\Vc}{\mathcal{V}}
\newcommand{\Wc}{\mathcal{W}}
\newcommand{\Xc}{\mathcal{X}}
\newcommand{\Yc}{\mathcal{Y}}
\newcommand{\Zc}{\mathcal{Z}}


\newcommand{\Sp}{\mathrm{Sp}}
\newcommand{\tr}{\mathrm{tr}}
\newcommand{\Op}{\mathrm{Op}}
\newcommand{\sym}{\mathrm{sym}}
\newcommand{\Vol}{\mathrm{Vol}}
\newcommand{\Tr}{\mathrm{Tr}}
\newcommand{\dist}{\mathrm{dist}}
\newcommand{\sgn}{\operatorname{sgn}}
\newcommand{\diag}{\mathrm{diag}}
\newcommand{\id}{\mathrm{id}}
\newcommand{\Poly}{\mathrm{Poly}}

\newcommand{\spec}{\mathrm{Spec}}
\newcommand{\abs}{\mathrm{abs}}

\newcommand{\CV}{\mathrm{CV}}
\newcommand{\PCV}{\mathrm{PCV}}


% Used for highlighting. To remove all highlighting just make the command blank
\newcommand{\hl}{\color{red}}



\newcommand{\dom}{\mathrm{dom}}
\newcommand{\Bl}{\mathbb{B}^4}
\newcommand{\supp}{\mathrm{supp}}
\newcommand{\BS}{\mathfrak{BS}}
\newcommand{\dyad}{\mathrm{dyad}}
\newcommand{\Qs}{\mathscr{Q}}
\newcommand{\Av}{\mathrm{Av}}
\newcommand{\loc}{\mathrm{loc}}
% DOI transformer
\newcommand{\Ti}{\mathcal{T}}
\newcommand{\sa}{\mathrm{sa}}


\newcommand{\Str}{\operatorname{Str}}
   
   
\newcommand{\gf}{\mathfrak{g}}

\date{8/3/2023}
   
\begin{document}

\title[Spectral asymptotics and nilpotent Lie groups]{Spectral asymptotics and scattering theory in the nilpotent Lie group setting}


\author[E. McDonald]{Edward McDonald\\
(based on joint work with  Z.~Fan, J.~Li, F.~Sukochev and D.~Zanin)}


\institute[]{Penn State University}

\makebeamertitle

\begin{frame}{Introduction}
    This talk is based on a series of preprints by myself with Zhijie Fan (Wuhan), Ji Li (Macquarie), Fedor Sukochev (UNSW) and Dmitriy Zanin (UNSW). 
    
    The first two papers are available:
    \begin{enumerate}[{\rm (a)}]
      \item{} Spectral estimates and asymptotics for stratified Lie groups \texttt{arXiv:2201.12349} (with Sukochev and Zanin)
      \item{} Endpoint weak Schatten class estimates and trace formula for commutators of Riesz transforms with multipliers on Heisenberg groups \texttt{arXiv:2201.12350} (with Fan, Li, Sukochev and Zanin) 
    \end{enumerate}
    There will also be other papers (currently in preparation).
\end{frame}

\begin{frame}{Plan for this talk}
    \begin{enumerate}
        \item{} Some elementary background on scattering theory
        \item{} Stratified lie groups and recent developments
        \item{} Singular values, Cwikel's estimates and Birman's theorem.
        \item{} Some new results
    \end{enumerate}
\end{frame}

\begin{frame}{Summary for the minister}
  In our preprints we have some technical results on the spectra of operators of the form
  \[
    M_f D^{-1}:L_2(G)\to L_2(G)
  \]
  where $G$ is a stratified Lie group, $M_f$ is the operator of pointwise multiplication by a function $f$ on $G$ and $D$ is a positive maximally hypoelliptic differential operator on $G.$

  Specifically, we now have a much better understanding of the singular values of these operators.
  \pause

  These results are interesting on their own, but I will discuss a program (mostly unrealized) to do scattering theory (in the style of Birman-Kato) for maximally hypoelliptic operators (in the style of of Helffer-Nourigat, Androulidakis-Mohsen-Yuncken).
\end{frame}

\begin{frame}{Summary for the minister (continued)}
Singular value estimates for operators like $M_fD^{-1}$ have several applications. For example:
\begin{itemize}
  \item{} Bound state problems: estimate the number of eigenvalues of $D+M_f.$
  \item{} Scattering theory: compare the effect of $M_f$ on the evolution of $\exp(it(D+M_f)),$
  \item{} Spectral theory: determine the Weyl asymptotics of general maximally hypoelliptic differential operators.
\end{itemize}
\end{frame}


\begin{frame}{Very elementary scattering theory}
  If $Q$ is an elliptic and symmetric differential operator
  \[
     Q:C^{\infty}(X,E)\to C^\infty(X,E)
  \]
  where $X$ is compact and Riemannian, and $E$ is some Hermitian vector bundle, then $Q$ is self-adjoint and has a discrete spectral decomposition
  \[
     Q = \sum_{n=0}^\infty \lambda(n,Q)P_n
  \]
  where $P_n$ is a finite rank $L_2(X,E)$-orthogonal projection, and $\{\lambda(n,Q)\}_{n=0}^\infty$ enumerates the spectrum of $Q$ in increasing order of absolute value.

  \pause

  If $X$ is not compact, this is of course not true.
\end{frame}

\begin{frame}{Very elementary scattering theory}
  Suppose that $X$ is not compact (later, we will simply take $X = \Rl^d$). If we assume that the geometry of $X$ and $E$ are not so bad and that the coefficients of $Q$ are uniformly bounded in the correct sense, then $Q$ is still self-adjoint but its spectrum is complicated.

  \pause
  Normally, we say that the spectral measure $E^Q$ of $Q$ splits into three mutually singular parts:
  \[
  E^{Q} = E^Q_{pp}+E^Q_{ac}+E^Q_{sc}
  \]
  the pure point spectrum (the eigenvalues), the absolutely continuous spectrum and the singular continuous spectrum.

  \pause
  In this case scattering theory can provide a more useful description than decomposing into eigenfunctions.
\end{frame}

\begin{frame}{Very elementary scattering theory}
  A very standard situation is that we have a symmetric differential operator (on $\Rl^d$),
  \[
      D_1 = \sum_{|\alpha|\leq m} a_{\alpha}(x)\partial^{\alpha}
  \]
  with smooth coefficients $\{a_{\alpha}\}$ that are constant outside of a compact set, say $a_{\alpha}(x) = c_{\alpha}.$
  \pause
  Consider the operator
  \[
      D_0 = \sum_{|\alpha|\leq m} c_{\alpha}\partial^{\alpha}
  \]
  \pause
  The spectral theory of $D_0$ is easy to understand using the Fourier transform: it is purely absolutely continuous.

  We expect that the absolutely continuous spectrum of $D_1$ somehow arises from that of $D_0.$
\end{frame}


\begin{frame}{Very elementary scattering theory}
  Scattering theory is about the solutions to the equation
  \[
    \frac{\partial u}{\partial t} = iD_1u.
  \]
  Or $u(t) = \exp(itD_1)u(0).$ We want to know when
  there exists $u_+$ such that
  \[
      \lim_{t\to\infty} \|\exp(itD_1)u(0)-\exp(itD_0)u_+\| = 0.
  \]
  Or, alternatively, when there exists a strong limit
  \[
    W_+(D_1,D_0) := {\mathrm{s}-}\lim_{t\to\infty} e^{-itD_0}e^{itD_1}.
  \]
  (actually, we are interested in a slight modification of this).
\end{frame}


\begin{frame}{Very elementary scattering theory}
  Let $D_0,D_1$ be self-adjoint operators on some Hilbert space $H,$ and let $P_{ac}(D_1)$ be the projection onto the absolutely continuous subspace of $D_1.$

  Define two operators $W_{\pm}(D_0,D_1)$ by
  \[
    W_{\pm}(D_0,D_1) := {\mathrm{s}-}\lim_{t\to\pm \infty} e^{-itD_0}e^{itD_1}P_{ac}(D_1).
  \]
  These are called the wave operators. We say that the wave operators (if they exist) are \emph{complete}
  if
  \[
      \mathrm{ran}(W_{\pm}(D_0,D_1)) = P_{ac}(D_0).
  \]
\end{frame}

\begin{frame}{Very elementary scattering theory}
  Here is the general picture to keep in mind. Suppose for the moment that $D_1$ does not have any singular continuous spectrum. We want to understand the solutions to the Schr\"odinger equation
  \[
    \frac{du}{dt}(t) = iD_1u(t),\quad u(0) = u_0.
  \]
  Splutting the initial value $u_0$ into the point and absolutely continuous parts, the solution looks like
  \[
    u(t) = \sum_{\lambda \in \mathrm{spec}_{pp}(D_1)} e^{it\lambda}E^{D_1}(\{\lambda\})u_0 + P_{ac}(D_1)u_0.
  \]
  If the wave operator $W_{+}(D_0,D_1)$ exists, then $P_{ac}(D_1)u_0$ looks asymptotically like a function evolving under $D_0.$
  \[
      \lim_{t\to\infty} \|e^{itD_0}u_+ - e^{itD_1}P_{ac}(D_1)u_0\| = 0
  \]
  where
  \[
      u_+ =W_+(D_0,D_1)u_0.
  \]
\end{frame}


\begin{frame}{Very elementary scattering theory}
  With a little more effort, we can compare the solutions of the wave equations
  \[
    \frac{\partial^2 u}{\partial t^2} = D_1u,\; \frac{\partial^2 u}{\partial t^2} = D_0u.
  \]
  (this is called acoustical scattering; see Reed-Simon Volume III.)
\end{frame}



\begin{frame}{Goals of scattering theory}
  As I see it, the primary goal of the Birman-Kato theory is to understand the absolutely continuous spectrum of an operator $D_1$ by relating it to a simpler operator $D_0.$
  If the wave operators $W_{+}(D_0,D_1)$ exists and is complete, then it provides a unitary equivalence between the absolutely continuous subspaces of $D_1$ and $D_0.$

  Another important task not directly related to scattering theory is to figure out how many eigenvalues there are in the point spectrum.
\end{frame}



\begin{frame}{Uses of scattering theory}
The Birman-Kato theory has had much application in geometry and topology. Some selected applications: \pause
\begin{itemize}
\item{} Relative index theorems: Suppose that $D_1$ and $D_0$ are odd self-adjoint operators on a $\Itgr/2\Itgr$-graded Hilbert space $H.$ The relative index of $D_1$ with respect to $D_0$ is
\[
\mathrm{ind}(D_1,D_0) = \mathrm{Str}(e^{-tD_1^2}-e^{-tD_0^2})
\]
(provided it exists). The relative index is the differences of the indices of $D_1$ and $D_0,$ plus an extra term coming from the continuous spectrum. See Eichhorn \emph{Relative Index Theory} (2008), and also Borisov-M\"uller-Schrader "Relative Index Theorems and Supersymmetric Scattering Theory" (1988).\pause
\end{itemize}
\end{frame}


\begin{frame}{Uses of scattering theory}
\begin{itemize}
\item{} Witten index: It is concievable that one could have a non-Fredholm operator $D$ such that
\[
\mathrm{wind}(D) := \lim_{t\to\infty}\Tr(e^{-tD^*D}-e^{-tDD^*})
\]
exists. This is called the Witten index, and can be expressed in terms of the scattering data of the pair $(|D|,|D^*|).$ See Carey-Gesztesy-Levitina-Sukochev "The spectral shift function and the Witten index" (2016).
\item{} The $K$-theoretical version of Levinson's theorem, see Richard, "Levinson’s Theorem: An Index Theorem in Scattering Theory" (2016)\pause
\end{itemize}
Closely related is the Lax-Phillips scattering theory, with its well-known applications in geometry (see Melrose, \emph{The Atiyah-Patodi-Singer index theorem} (1992), Lax-Phillips \emph{Scattering theory} (1989)).

\end{frame}


\begin{frame}{Birman's theorem}
  Suppose that $A_1,A_0$ are self-adjoint operators on a Hilbert space $H.$ If for any bounded interval $I\subset \Rl$ we have
  \[
    E^{A_1}(I)(A_1-A_0)E^{A_0}(I) \in \Lc_1(H)
  \]
  then the wave operators $W_{\pm}(A_1,A_0)$ exist and are complete.\\\pause

  {\color{red} There are some additional technical assumptions on $A_1,A_0$ which I will not address here.}\\ \pause

  It suffices, for example, to have
  \[
    (A_1-A_0)(1+A_0^2)^{-N} \in \Lc_1(H)
  \]
  for sufficiently large $N.$
\end{frame}

\begin{frame}{Using Birman's theorem}
  Consider the pair
  \[
      A_1 =c(x)\Delta, A_0 = \Delta = \sum_{j=1}^d \partial_{x_j}^2
  \]
  on $\Rl^d,$ where $c$ is a smooth positive function equal to $1$ outside a compact set. Then
  \[
    (A_1-A_0)(1-\Delta)^{-N} =(c(x)-1)\Delta (1-\Delta)^{-N}
  \]
  This belongs to $\Lc_1$ for sufficiently large $N,$ thanks to some old results of Birman-Solomyak.
\end{frame}


\begin{frame}{Stratified Lie groups}
  Let $\gf$ be a Lie algebra which admits a direct sum decomposition
  \[
    \gf = \bigoplus_{n=1}^\infty \gf_n
  \]
  where $[\gf_k,\gf_n] \subseteq \gf_{k+n}$ and $\gf_1$ generates $\gf.$
  This is called a stratified Lie algebra.

  The number
  \[
    Q := \sum_{n=1}^\infty n\mathrm{dim}(\gf_n)
  \]
  is called the homogeneous dimension of $\gf.$
\end{frame}

\begin{frame}{Stratified Lie groups}
  Exponentiating $\gf,$ we get a simply connected nilpotent Lie group
  \[
    G = \exp(\gf).
  \]
  This is a homeomorphism, and the Lebesgue measure of $\gf$ pushes forward to the Haar measure of $G.$
  Suppose that $\gf_1$ has a basis $\{X_1,\ldots,X_m\}.$, and $G$ is essentially a Euclidean space $\Rl^d$ equipped with a family of vector fields
  \[
    X_1,\ldots,X_m
  \]
  with polynomial coefficients satisfying the H\"ormander condition at every point.
\end{frame}

\begin{frame}{Ellipticity on stratified Lie groups}
  The stratification of $\gf$ defines a grading on the algebra of invariant differential operators, $\Uc(\gf),$ on $G.$
  Say that an operator $P\in \Uc(\gf)$ has order $k$ if the highest degree term in $P$ is homogeneous of degree $k.$

  \begin{theorem}[Helffer-Nourigat, Rockland]
    Let $P \in \Uc(\gf)$ have degree $k.$ If for every $\pi \in \widehat{G}_u$ (the unitary dual of $G$), $\pi(P)$
    is injective on $H^{\infty}_\pi$ (the smooth vectors), then for every $Q$ of degree less than or equal to $k$ we have
    \[
      \|Qu\|_{L_2(G)} \lesssim \|Pu\|_{L_2(G)}+\|u\|_{L_2(G)},\quad u\in L_2(G).
    \]
    In particular, $P$ is hypoelliptic.
  \end{theorem}
\end{frame}

\begin{frame}{Ellipticity on stratified Lie groups}
  Recently some substantial advances have been made in the study of ellipticity on Heisenberg manifolds and more general foliated manifolds.
  \pause
  This opens up the possibility to study the scattering theory for maximally hypoelliptic operators.
\end{frame}

\begin{frame}{Some results}
Recall that $\{X_1,\ldots,X_m\}$ denotes a basis for $\gf_1,$ the first layer of our stratified Lie algebra. By assumption $X_1,X_2,\ldots,X_m$ generate $\gf.$ Let
\[
  \Delta := \sum_{j=1}^m X_j^2.
\]
This is hypoelliptic.\\
\pause
Given a function $f$ on $G,$ denote by $M_f$ the (possibly unbounded) operator of pointwise multiplication by $f.$ We want to understand the operators
\[
  M_f(1-\Delta)^{-N},\quad (1-\Delta)^{-N}M_f(1-\Delta)^{-N}
\]
and their trace ideal properties.\\
\pause
Why is this?\\
\pause
Among other things, so we can use Birman's theorem to study the scattering of differential operators on $G.$
\end{frame}

\begin{frame}{A first result}
One not-entirely-trivial results we obtained is the following.
\begin{theorem}[M.-Sukochev-Zanin]
  Let $r>Q$ (recall that $Q$ is the homogeneous dimension) and let $q>2.$ Given $f \in \ell_1(L_q)(G)$ {\color{red}(a function space on $G$)}, the operator
  \[
M_f(1-\Delta)^{-\frac{r}{2}}:L_2(G)\to L_2(G)
  \]
  is trace class.
\end{theorem}
\end{frame}

\begin{frame}{Reminder on singular values}
Given a compact operator $T$ on some Hilbert space, the $(n+1)$-st singular value of $T$ is defined as
\[
  \mu(n,T) :=\inf\{\|T-R\|\;:\;\mathrm{rank}(R)\leq n\}.
\]
One say that $T \in \Lc_{p,\infty}(H)$ if $\mu(n.T) = O(n^{-\frac1p}),$ with
\[
  \|T\|_{p,\infty} := \sup_{n\geq 0} (n+1)^{\frac1p}\mu(n,T).
\]
\end{frame}

\begin{frame}{Some results}
\begin{theorem}\label{main_nontrivial_cwikel_theorem}
Let $G$ be a stratified Lie group with stratification $\gf = \bigoplus_{n=1}^\infty \gf_n,$ homogeneous dimension $Q = \sum_{n=1}^\infty n\cdot \mathrm{dim}(\gf_n)$
and a fixed sub-Laplacian $\Delta = \sum_{j=1}^m X_j^2,$ where $\{X_j\}_{j=1}^m$ is a basis for $\gf_1.$
\begin{enumerate}[{\rm (i)}]
\item\label{mncta} if $p>2,$ then
$$\|M_f(-\Delta)^{-\frac{Q}{2p}}\|_{p,\infty}\leq c_p\|f\|_{L_p(G)}$$
\item\label{mnctb} if $p<2$ and $q>2,$ then
$$\|M_f(1-\Delta)^{-\frac{Q}{2p}}\|_{p,\infty}\leq c_{p,q}\|f\|_{\ell_p(L_q)(G)}.$$
\item\label{mnctc} if $p=2$ and $q>2,$ then
$$\|M_f(1-\Delta)^{-\frac{Q}{2p}}\|_{p,\infty}\leq c_q\|f\|_{\ell_{2,\log}(L_q)(G)}.$$
\end{enumerate}
\end{theorem}
% Having Theorem \ref{main_nontrivial_cwikel_theorem} at hands and using H\"older's inequality, we immediately obtain the following corollary.
%
% \begin{corollary}\label{main_nontrivial_cwikel_corollary} Let $G$ be a stratified Lie group with stratification $\gf = \bigoplus_{n=1}^\infty \gf_n,$ homogeneous dimension $Q = \sum_{n=1}^\infty n\cdot \mathrm{dim}(\gf_n)$
% and a fixed sub-Laplacian $\Delta = \sum_{j=1}^m X_j^2,$ where $\{X_j\}_{j=1}^m$ is a basis for $\gf_1.$
% \begin{enumerate}[{\rm (i)}]
% \item if $p>1,$ then
% $$\|(-\Delta)^{-\frac{Q}{4p}}M_f(-\Delta)^{-\frac{Q}{4p}}\|_{p,\infty}\leq c_p\|f\|_{L_p(G)}.$$
% \item if $p<1$ and $q>1,$ then
% $$\|(1-\Delta)^{-\frac{Q}{4p}}M_f(1-\Delta)^{-\frac{Q}{4p}}\|_{p,\infty}\leq c_{p,q}\|f\|_{\ell_p(L_q)(G)}.$$
% \item if $p=1$ and $q>1,$ then
% $$\|(1-\Delta)^{-\frac{Q}{4p}}M_f(1-\Delta)^{-\frac{Q}{4p}}\|_{p,\infty}\leq c_q\|f\|_{\ell_{1,\log}(L_q)(G)}.$$
% \end{enumerate}
% \end{corollary}
\end{frame}


\begin{frame}{Some results}
Of course, a similar result holds for Schatten ideals.

\begin{theorem}\label{cwikel schatten_theorem}
\begin{enumerate}[{\rm (i)}]
\item if $p>2$ and $r>\frac{Q}{p},$ then
$$\|M_f(-\Delta)^{-\frac{r}{2}}\|_p\leq c_{p,r}\|f\|_{L_p(G)}.$$
\item if $p=2$ and $r>\frac{Q}{p},$ then
$$\|M_f(1-\Delta)^{-\frac{r}{2}}\|_p=c_{p,r}\|f\|_{L_p(G)}.$$
\item if $p<2,$ $r>\frac{Q}{p}$and $q>2,$ then
$$\|M_f(1-\Delta)^{-\frac{r}{2}}\|_p\leq c_{p,q,r}\|f\|_{\ell_p(L_q)(G)}.$$
\end{enumerate}
\end{theorem}
\end{frame}


\begin{frame}{Birman's theorem for stratified Lie groups}
  Suppose that
  \[
      D_1 = \sum_{|\alpha|_h\leq m} a_{\alpha}(x)X^{\alpha}
  \]
  where each $a_{\alpha}$ is a smooth function on $G$ equal to a constant (say, $c_{\alpha}$) outside a compact set.
  Then we expect that
  \[
      D_0 = \sum_{|\alpha|_h\leq m} c_{\alpha}X^{\alpha}
  \]
  is a good model for $D_1$ asymptotically, since $D_1-D_0$ is a differential operator with compactly supported coefficients.

  The preceding theorems verify Birman's theorem for $D_1,D_0.$
\end{frame}

\begin{frame}{What about the point spectrum?}
  These estimates are also useful to estimate the number of eigenvalues of operators.
\begin{theorem}[Cwikel--Lieb--Rozenblum estimate]\label{CLR_theorem}
    Assume that $Q>2.$ Let $V \in L_{\frac{Q}{2}}(G)$ be real-valued. The quadratic form sum
    \[
        -\Delta\dot{+}M_V
    \]
    is well-defined on the form domain $W^1_2(G),$ and defines an unbounded self-adjoint operator on $L_2(G)$ with essential spectrum $[0,\infty).$ The operator $-\Delta\dot{+}M_V$ has finitely many negative eigenvalues, and the total number of eigenvalues less than $-t$ for $t\geq 0$ is bounded by
    \[
        \Tr(\chi_{(-\infty,-t)}(-\Delta\dot{+}M_V)) \leq C_{G}\int_{G} (V+t)_-^{\frac{Q}{2}}.
    \]
\end{theorem}
\end{frame}




\begin{frame}{Spectral asymptotics}
Related to these estimates we have spectral asymptotics. In the following theorem, $\mu$ denotes the singular value function. In particular, the sequence $\{\mu(n,T)\}_{n=0}^\infty$ is the sequence of singular values of a compact operator $T.$ We give a precise definition of $\mu$ in the next section.
\begin{theorem}\label{main_asymptotic_formula}
Let $G$ be a non-abelian stratified Lie group with stratification $\gf = \bigoplus_{n=1}^\infty \gf_n,$ homogeneous dimension $Q = \sum_{n=1}^\infty n\cdot \mathrm{dim}(\gf_n)$
and a fixed sub-Laplacian $\Delta = \sum_{j=1}^m X_j^2,$ where $\{X_j\}_{j=1}^m$ is a basis for $\gf_1.$ Let $k\in\mathbb{N}$ and let $p=\frac{Q}{k}.$
% If one of the following conditions holds
% \begin{enumerate}[{\rm (i)}]
% \item{} $p>1$ and $0\leq f \in L_p(G);$
% \item{} $p<1$ and $0\leq f\in \ell_p(L_q)(G)$ for some $q>1;$
% \item{} $p=1$ and $0\leq f\in \ell_{1,\log}(L_q)(G)$ for some $q>1;$
% \end{enumerate}
then {\color{red} Under some technical assumptions on $f$ (depending on $p$),}
then there exists the limit
$$\lim_{t\to\infty} t\mu(t,(1-\Delta)^{-\frac{k}{4}}M_f(1-\Delta)^{-\frac{k}{4}})^p=c_G\int_G f^p.$$
% Moreover, if $k<Q,$ then we also have the existence of the limit
% $$\lim_{t\to\infty} t\mu(t,(-\Delta)^{-\frac{k}{4}}M_f(-\Delta)^{-\frac{k}{4}})^p=c_G\int_G f^p.$$
Here, the constant $c_G>0$ depends on the stratification and also on the particular choice of the basis in $\gf_1.$
\end{theorem}
\end{frame}


\begin{frame}{Semiclassical corollary}
\begin{corollary}
    Let $G\neq\mathbb{H}^1$ be a stratified Lie group with stratification $\gf = \bigoplus_{n=1}^\infty \gf_n,$ homogeneous dimension $Q = \sum_{n=1}^\infty n\cdot \mathrm{dim}(\gf_n)$
    and a fixed sub-Laplacian $\Delta = \sum_{j=1}^m X_j^2,$ where $\{X_j\}_{j=1}^m$ is a basis for $\gf_1.$

    Assume that $V\in L_{\frac{Q}{2}}(G)$ is real-valued. For $h>0,$ the operator $-h^2\Delta\dot{+}M_V$ can be defined in the sense of quadratic forms.
    There exists a constant $c_{G}>0$ such that
    \[
        \lim_{h\to 0} h^{Q}\Tr(\chi_{(-\infty,0)}(-h^2\Delta\dot{+}M_V)) = c_{G}\int_{G} V_-^{\frac{Q}{2}}.
    \]
    Here, $V_- = \frac{1}{2}(|V|-V)$ is the negative part of $V.$
\end{corollary}
\end{frame}




\begin{frame}{The future}
These estimates are suboptimal for a number of reasons, one of them being that we state the results for functions on $G$ rather than a general Heisenberg manifold (or an even more general filtered manifold). This is probably not a significant restriction.
\end{frame}



\begin{frame}
\structure{\begin{center}
{\Huge{}Thank you for listening!}
\par\end{center}}\end{frame}



\end{document}

