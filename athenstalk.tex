\documentclass{beamer}



%\usepackage{beamerthemesplit}
\usetheme{Boadilla}
%\usetheme{default}
%\useinnertheme{rounded}

%\useoutertheme{shadow}
\usecolortheme{rose}
%\usefonttheme{serif}
\setbeamertemplate{navigation symbols}{}
\usetheme{Madrid}

\usepackage{amssymb,amsmath,amscd,amsfonts,amsthm,dsfont,color,graphicx}
\usepackage{amscd}
%\usepackage[numbers]{natbib}
% \usepackage[french]{babel}
%\usepackage[active]{srcltx}


\def\qd{\,{\mathchar'26\mkern-12mu d}}

% \date[]{}

 \newcommand\makebeamertitle{\frame{\maketitle}}%

 \AtBeginDocument{
   \let\origtableofcontents=\tableofcontents
   \def\tableofcontents{\@ifnextchar[{\origtableofcontents}{\gobbletableofcontents}}
   \def\gobbletableofcontents#1{\origtableofcontents}
 }
\numberwithin{equation}{section}
  \theoremstyle{plain}
  \newtheorem*{thm*}{\protect\theoremname}
  \theoremstyle{plain}
  \newtheorem*{cor*}{\protect\corollaryname}
 \theoremstyle{definition}
 \newtheorem*{defn*}{\protect\definitionname}
 \theoremstyle{plain}
\newtheorem*{lem*}{\protect\lemmaname}
  \theoremstyle{plain}
  \newtheorem*{rem*}{\protect\remarkname}
   \theoremstyle{definition}
 \newtheorem*{prop*}{\protect\propositionname}

\usetheme{Madrid}

\makeatother

  \providecommand{\corollaryname}{Corollary}
  \providecommand{\definitionname}{Definitioninition}
  \providecommand{\theoremname}{Theorem}
   \providecommand{\lemmaname}{Lemma}
   \providecommand{\remarkname}{Remark}
   \providecommand{\propositionname}{Proposition}
   
   
\newcommand{\Rl}{\mathbb{R}}
\newcommand{\Cplx}{\mathbb{C}}
\newcommand{\Itgr}{\mathbb{Z}}
\newcommand{\Ntrl}{\mathbb{N}}
\newcommand{\Circ}{\mathbb{T}}
\newcommand{\Sb}{\mathbb{S}}
\newcommand{\Disc}{\mathbb{D}}
\newcommand{\Aff}{\mathbb{A}}

% The Caligraphic alphabet
\newcommand{\Ac}{\mathcal{A}}
\newcommand{\Bc}{\mathcal{B}}
\newcommand{\Cc}{\mathcal{C}}
\newcommand{\Dc}{\mathcal{D}}
\newcommand{\Ec}{\mathcal{E}}
\newcommand{\Fc}{\mathcal{F}}
\newcommand{\Gc}{\mathcal{G}}
\newcommand{\Hc}{\mathcal{H}}
\newcommand{\Ic}{\mathcal{I}}
\newcommand{\Jc}{\mathcal{J}}
\newcommand{\Kc}{\mathcal{K}}
\newcommand{\Lc}{\mathcal{L}}
\newcommand{\Mv}{\mathcal{M}}
\newcommand{\Nv}{\mathcal{N}}
\newcommand{\Oc}{\mathcal{O}}
\newcommand{\Pc}{\mathcal{P}}
\newcommand{\Qc}{\mathcal{Q}}
\newcommand{\Rc}{\mathcal{R}}
\newcommand{\Sc}{\mathcal{S}}
\newcommand{\Tc}{\mathcal{T}}
\newcommand{\Uc}{\mathcal{U}}
\newcommand{\Vc}{\mathcal{V}}
\newcommand{\Wc}{\mathcal{W}}
\newcommand{\Xc}{\mathcal{X}}
\newcommand{\Yc}{\mathcal{Y}}
\newcommand{\Zc}{\mathcal{Z}}


\newcommand{\Sp}{\mathrm{Sp}}
\newcommand{\tr}{\mathrm{tr}}
\newcommand{\Op}{\mathrm{Op}}
\newcommand{\sym}{\mathrm{sym}}
\newcommand{\Vol}{\mathrm{Vol}}
\newcommand{\Tr}{\mathrm{Tr}}
\newcommand{\dist}{\mathrm{dist}}
\newcommand{\sgn}{\operatorname{sgn}}
\newcommand{\diag}{\mathrm{diag}}
\newcommand{\id}{\mathrm{id}}
\newcommand{\Poly}{\mathrm{Poly}}

\newcommand{\spec}{\mathrm{Spec}}
\newcommand{\abs}{\mathrm{abs}}

\newcommand{\CV}{\mathrm{CV}}
\newcommand{\PCV}{\mathrm{PCV}}


% Used for highlighting. To remove all highlighting just make the command blank
\newcommand{\hl}{\color{red}}



\newcommand{\dom}{\mathrm{dom}}
\newcommand{\Bl}{\mathbb{B}^4}
\newcommand{\supp}{\mathrm{supp}}
\newcommand{\BS}{\mathfrak{BS}}
\newcommand{\dyad}{\mathrm{dyad}}
\newcommand{\Qs}{\mathscr{Q}}
\newcommand{\Av}{\mathrm{Av}}
\newcommand{\loc}{\mathrm{loc}}
% DOI transformer
\newcommand{\Ti}{\mathcal{T}}
\newcommand{\sa}{\mathrm{sa}}


\newcommand{\Str}{\operatorname{Str}}
   
   
\newcommand{\gf}{\mathfrak{g}}

\date{\today}
   
\begin{document}

\title[Spectral asymptotics and nilpotent Lie groups]{Spectral asymptotics and scattering theory in the nilpotent Lie group setting}


\author[E. McDonald]{Edward McDonald\\
(based on joint work with  Z.~Fan, J.~Li, F.~Sukochev and D.~Zanin)}


\institute[]{Penn State University}

\makebeamertitle

\begin{frame}{Introduction}
    This talk is based on a series of preprints by myself with Zhijie Fan (Wuhan), Ji Li (Macquarie), Fedor Sukochev (UNSW) and Dmitriy Zanin (UNSW). 
    
    The first two papers are available:
    \begin{enumerate}[{\rm (a)}]
      \item{} Spectral estimates and asymptotics for stratified Lie groups \texttt{arXiv:2201.12349} (with Sukochev and Zanin)
      \item{} Endpoint weak Schatten class estimates and trace formula for commutators of Riesz transforms with multipliers on Heisenberg groups \texttt{arXiv:2201.12350} (with Fan, Li, Sukochev and Zanin) 
    \end{enumerate}
    There will also be other papers (currently in preparation).
\end{frame}

\begin{frame}{Plan for this talk}
    \begin{enumerate}
        \item{} Some elementary background on scattering theory (why do we care?)
        \item{} Stratified lie groups and recent developments
        \item{} The future(?)
    \end{enumerate}
    I will discuss a program (mostly unfinished) to do scattering theory (in the style of Birman-Kato) for maximally hypoelliptic operators (in the style of of Helffer-Nourigat, Androulidakis-Mohsen-Yuncken).
\end{frame}

\begin{frame}{Very elementary scattering theory}
  If $Q$ is an elliptic and symmetric differential operator
  \[
     Q:C^{\infty}(X,E)\to C^\infty(X,E)
  \]
  where $X$ is compact and Riemannian, and $E$ is some Hermitian vector bundle, then $Q$ is self-adjoint and has a discrete spectral decomposition
  \[
     Q = \sum_{n=0}^\infty \lambda(n,Q)P_n
  \]
  where $P_n$ is a finite rank $L_2(X,E)$-orthogonal projection, and $\{\lambda(n,Q)\}_{n=0}^\infty$ enumerates the spectrum in increasing order of absolute value. 
    
  \pause
  
  If $X$ is not compact, this is of course not true.
\end{frame}

\begin{frame}{Very elementary scattering theory}
  Suppose that $X$ is not compact (later, we will simply take $X = \Rl^d$). If we assume that the geometry of $X$ and $E$ are not so bad and that the coefficients of $Q$ are uniformly bounded in the correct sense, then $Q$ is still self-adjoint but its spectrum is complicated.
  
  \pause
  In this case scattering theory can provide a more useful description. 
\end{frame}

\begin{frame}{Very elementary scattering theory}
  A conventional situation is that we have a differential operator (on $\Rl^d$),
  \[
      D_1 = \sum_{|\alpha|\leq m} a_{\alpha}(x)\partial^{\alpha}
  \]
  with smooth coefficients $\{a_{\alpha}\}$ that are constant outside of a compact set, say $a_{\alpha}(x) = c_{\alpha}.$ If we define
  \[
      D_0 = \sum_{|\alpha|\leq m} c_{\alpha}\partial^{\alpha}
  \]
  The spectral theory of $D_0$ is easy to understand using the Fourier transform. We expect that $D_1$
  
  
  Scattering theory is about the solutions to the equation
  \[
    \frac{\partial u}{\partial t} = iD_1u.
  \]
  Or $u(t) = \exp(itD_1)u(0).$ We want to know when
  there exists $u_+$ such that
  \[
      \lim_{t\to\infty} \|\exp(itD_1)u(0)-\exp(itD_0)u_+\| = 0.
  \] 
  Or, alternatively, when there exists a strong limit
  \[
    W_+(D_1,D_0) := {\mathrm{s}-}\lim_{t\to\infty} e^{-itD_0}e^{itD_1}.
  \]
  (actually, we are interested in a slight modification of this).
\end{frame}

\begin{frame}{Stratified Lie groups}
  Let $\gf$ be a Lie algebra which admits a direct sum decomposition
  \[
    \gf = \bigoplus_{k=1}^\infty \gf_k
  \]
  where $[\gf_k,\gf_l] \subseteq \gf_{k+l}$ and $\gf_1$ generates $\gf.$ 
  This is called a stratified Lie algebra. Exponentiating $\gf,$
  we get a simply connected nilpotent Lie group
  \[
    G = \exp(\gf).
  \]
  This is a homeomorphism, and the Lebesgue measure of $\gf$ pushes forward to the Haar measure of $G.$
  Suppose that $\gf_1$ has a basis $\{X_1,\ldots,X_m\}.$, and $G$ is essentially a Euclidean space $\Rl^d$ equipped with a family of vector fields
  \[
    X_1,\ldots,X_m
  \]
  with polynomial coefficients satisfying the H\"ormander condition at every point.
\end{frame}

\begin{frame}{Ellipticity on stratified Lie groups}
  The stratification of $\gf$ defines a grading on the algebra of invariant differential operators, $\Uc(\gf),$ on $G.$
  Say that an operator $P\in \Uc(\gf)$ has order $k$ if the highest degree term in $P$ is homogeneous of degree $k.$

  \begin{theorem}[Helffer-Nourigat, Rockland]
    Let $P \in \Uc(\gf)$ have degree $k.$ If for every $\pi \in \widehat{G}_u$ (the unitary dual of $G$), $\pi(P)$
    is injective on $H^{\infty}_\pi$ (the smooth vectors), then for every $Q$ of degree less than or equal to $k$ we have
    \[
      \|Qu\|_{L_2(G)} \lesssim \|Pu\|_{L_2(G)}+\|u\|_{L_2(G)},\quad u\in L_2(G).
    \]
    In particular, $P$ is hypoelliptic.
  \end{theorem}
\end{frame}

\begin{frame}{Ellipticity on stratified Lie groups}
\end{frame}


\begin{frame}{Birman's theorem}
  Suppose that $A_1,A_0$ are self-adjoint operators on a Hilbert space $H.$ If for any bounded interval $I\subset \Rl$ we have
  \[
    \chi_I(A_1)(A_1-A_0)\chi_{I}(A_0) \in \Lc_1(H)
  \]
  then the wave operators $W_{\pm}(A_1,A_0)$ exist and are complete.
\end{frame}

\begin{frame}{Birman's theorem for stratified Lie groups}
  Suppose that 
  \[
      D_1 = \sum_{|\alpha|_h\leq m} a_{\alpha}(x)X^{\alpha}
  \]
  where each $a_{\alpha}$ is a smooth function on $G$ equal to a constant (say, $c_{\alpha}$) outside a compact set.
  Then we expect that
  \[
      D_0 = \sum_{|\alpha|_h\leq m} c_{\alpha}X^{\alpha}
  \]
  is a good model for $D_1$ asymptotically, since $D_1-D_0$ is a differential operator with compactly supported coefficients.  
\end{frame}

\begin{frame}{A first result}
  \begin{theorem}
    Let $f \in C^\infty_c(G),$ and let $Q\in \Uc(\gf)$ be a Rockland operator. Then for any bounded interval $I\subset \Rl,$ we have
    \[
      M_f\chi_{I}(Q) \in \Lc_1(L_2(G)).
    \]
  \end{theorem}

\end{frame}


\begin{frame}
\structure{\begin{center}
{\Huge{}Thank you for listening!}
\par\end{center}}\end{frame}



\end{document}

